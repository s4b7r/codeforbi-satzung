\documentclass[12pt,a4paper,draft]{article}
\usepackage[ngerman]{babel}
\usepackage[T1]{fontenc}
\usepackage[utf8]{inputenc}

\newcommand{\unsername}{Code for Niederrhein}

\renewcommand{\labelenumi}{(\arabic{enumi})}
\renewcommand{\labelenumii}{\arabic{enumii}.}

\title{Beitragsordnung des Vereins \unsername}
\author{\unsername}
\date{\today}

\begin{document}

\maketitle
\tableofcontents

\section{Allgemeines}
\begin{enumerate}
\item Diese Beitragsordnung ist nicht Bestandteil der Satzung. Sie regelt die 
Beitragsverpflichtungen der Mitglieder sowie die Gebühren und Umlagen. Sie kann 
nur von der Mitgliederversammlung des Vereins geändert werden.

\item Die Mitgliederversammlung beschließt die Höhe der Geldbeiträge, die von 
Mitgliedern erhoben werden.

\item Von ordentlichen Mitgliedern und Fördermitgliedern werden Geldbeiträge 
erhoben.

\item Von Ehrenmitgliedern werden keine Geldbeiträge erhoben.

\item Die Beitragsordnung tritt mit Beschluss durch die Mitgliederversammlung in Kraft.
\end{enumerate}

\section{Regelmäßige Beiträge}
\begin{enumerate}
\item Es kann eine beitragsfreie Aufnahme (0 EUR Mindestbeitrag) von ordentlichen Mitgliedern erfolgen.

\item Von Fördermitgliedern wird ein jährlicher Mindestbeitrag von 24 EUR erhoben.

\item Über den Mindestbeitrag hinaus entscheidet jedes Mitglied selbst über die Höhe des jährlichen
finanziellen Beitrags.

\item Änderungswünsche zum individuellen, jährlichen finanziellen Beitrag werden in Textform formlos an
ein Vorstandsmitglied gegeben. Änderungswünsche für den Beitrag im kommenden Jahr können bis zum 30.11.
des aktuellen Jahres abgegeben werden. Eine Änderung des Beitrags für das aktuelle Jahr ist nicht möglich.

\item Zum Anfang eines jeden Kalenderjahres oder mit Beitritt in den Verein wird der gesamte Beitrag für
das entsprechende Kalenderjahr fällig.

\item Die Beiträge werden per SEPA-Lastschriftverfahren eingezogen. Jedes Mitglied erteilt dem Verein
dazu mit der Beitrittserklärung oder bei Änderung des persönlichen Beitrags das Lastschriftmandat.

\item Bei Beendigung der Mitgliedschaft werden keine Mitgliedsbeiträge 
erstattet.
\end{enumerate}

\section{Umlagen}

\item Von ordentlichen und Fördermitgliedern können Umlagen erhoben werden.

\item Umlagen können nur erhoben werden, wenn dies zur geregelten Abwicklung von (technischen) Einrichtungen 
des Vereins notwendig ist, die ohne die Umlage nicht länger finanziert werden könnten. Dies ist beispielsweise
gegeben, wenn ... Umlage vllt für den akuten Erhalt technischer Betriebsmittel oder sowas, sprich:
Server (Dienstleistungen) etc., um gezielte Abwicklung zu ermöglichen, Datenverlust vorzubeugen,
ggf. vorher geeignet nach außen kommunizieren zu können, etc.

\item Von einem Mitglied dürfen jährlich, in Summe höchstens Umlagen in Höhe eines Jahresbeitrags des
Mitglieds erhoben werden, mindestens jedoch 10 EUR.

\item Über die Festsetzung von Umlagen entscheidet der Vorstand.


\end{document}